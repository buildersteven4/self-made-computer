\documentclass[oneside, a4paper]{memoir}

\begin{document}
% Title
\title{SGPC Programmer's Manual}
\author{Steven Vroom}
\date{November 2016}
\maketitle
\cleardoublepage

% Front Matter 
\frontmatter
\setcounter{tocdepth}{2}
\tableofcontents
\cleardoublepage

\chapter{About This Manual}
\section{Related Documentation}
\section{Organization}
\section{Conventions}
\section{Acronyms and Abbreviations}

% Main Matter
\mainmatter
\chapter{Overview}
\section{SGPC Architecture Overview}
\section{Registers}
\subsection{User-accessible Registers}
\subsection{Internal Registers}
\section{Instruction Conventions}
\subsection{Instruction Layout}
\subsection{Addressing Modes}
\section{Instruction Set}
\section{Interrupt Model}
\section{Memory Management Model}

\chapter{Register Set}
This chapter describes the registers. The registers are seperated in four groups based on accessability.
\section{Foreground Registers}
The foreground registers are the registers all regular instructions can read from and write to. There are eight 8-bit and eight 16-bit foreground registers. These registers are preserved in interrupts.
\begin{table}[]
\centering
\caption{List of Foreground Registers}
\label{List of Foreground Registers}
\begin{tabular}{|l|l|l|l|}
\hline
ID  & Mnonic & Descriptive Name      & Length in bits \\ \hline
0x0 & al     & The lower byte of ax  & 8              \\ \hline
0x1 & ah     & The higher byte of ax & 8              \\ \hline
0x2 & bl     & The lower byte of bx  & 8              \\ \hline
0x3 & bh     & The higher byte of bx & 8              \\ \hline
0x4 & cl     & The lower byte of cx  & 8              \\ \hline
0x5 & ch     & The higher byte of cx & 8              \\ \hline
0x6 & dl     & The lower byte of dx  & 8              \\ \hline
0x7 & dh     & The higher byte of dx & 8              \\ \hline
0x8 & ax     & The first GPR         & 16             \\ \hline
0x9 & bx     & The second GPR        & 16             \\ \hline
0xA & cx     & The third GPR         & 16             \\ \hline
0xB & dx     & The fourth GPR        & 16             \\ \hline
0xC & ex     & The fifth GPR         & 16             \\ \hline
0xD & tm     & Temporary data        & 16             \\ \hline
0xE & sp     & Stack pointer         & 16             \\ \hline
0xF & pc     & Program counter       & 16             \\ \hline
\end{tabular}
\end{table}
\subsection{General-Purpose Registers (GPRs)}
These instruction
\subsection{Stack Pointer Register (SP)}
\subsection{Program Counter Register (PC)}
\section{Background Registers}
\subsection{Backup Registers}
\subsection{Interrupt Registers}
\section{Indirect Registers}
\subsection{Flags Register}
\subsection{Section Registers}
\section{Output Registers}

\chapter{Operand Conventions and Addressing Modes}
\section{Operand Conventions}
\subsection{Bit and Byte Ordering}
\subsection{Aligned and Misaligned Memory Access}
\section{Addressing Modes}
\subsection{Not From Memory}
\subsubsection{Register Direct}
\subsubsection{Absolute}
\subsubsection{Register with displacement}
\subsection{From Memory}
\subsubsection{Direct}
\subsubsection{Base Plus Displacement}

\chapter{Instruction Set Summary}
\section{Instruction Types}
\subsection{Move Instructions}
\subsection{Arithimic Instructions}
\subsection{Control Instructions}
\subsection{Reserved Instructions}
\subsection{Artifact Instructions}
\section{Instruction Format}
\subsection{exceptions}

\chapter{Memory Management}
\section{Segments}
\subsection{Base}
\subsection{Limit}
\section{Code Segment (CS)}
\section{Data Segment (DS)}
\section{Segment Switching}
\section{Changing Segments}

\chapter{Interrupts}
\section{Interrupt Enabling/Disabling}
\subsection{Interrupt Enabled Flag}
\subsection{Internal Interrupt Mask}
\subsection{Programmable Interrupt Controller (PIC)}
\section{State Preservation}
\subsection{Backup}
\subsection{Full Restore}
\subsection{Partial Restore}
\section{Interrupt Service Routine (ISR)}
\subsection{Interrupt Far Jump}
\subsection{Return to Context}
\subsection{Switch Context}

\chapter{I/O Conventions}
\section{Reading Input}
\section{Writing Output}
\subsection{Problem with Interrupts}

\chapter{Instruction Set}
\section*{0x00: MOVZ}
\section*{0x01: MOVNZ}
\section*{0x02: MOVS}
\section*{0x03: MOVNS}
\section*{...}

% Appendix
\appendix
%\backmatter
\chapter{Instruction Set Listings}

\chapter{Simplified Mnemonics}

\chapter{Common Procedures}

\chapter{Standard Peripherals}
\section{Programmable Interrupt Controller (PIC)}
\section{Keyboard}
\section{Programmable Interrupt Timer (PIT)}
\section{Sound Card}
\section{Graphical Card}
\section{Memory Control Hub (MCH)}
\section{Segements and Out Of Bounds Exception}

\end{document}
