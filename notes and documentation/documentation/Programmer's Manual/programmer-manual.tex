\documentclass[oneside, a4paper]{memoir}

\usepackage{multirow}
\usepackage{longtable}
\usepackage[table]{xcolor}
%\usepackage{hhline}

\begin{document}
% Title
\title{SGPC Programmer's Manual}
\author{Steven Vroom}
\date{November 2016}
\maketitle
\cleardoublepage

\setlength\arrayrulewidth{1pt}
\rowcolors{2}{}{gray!20}

% Front Matter 
\frontmatter
\setcounter{tocdepth}{2}
\tableofcontents
\cleardoublepage

\chapter{About This Manual}
\section{Related Documentation}
\section{Organization}
\section{Conventions}
\section{Acronyms and Abbreviations}

% Main Matter
\mainmatter
\chapter{Overview}
\section{SGPC Architecture Overview}
\section{Registers}
\subsection{User-accessible Registers}
\subsection{Internal Registers}
\section{Instruction Conventions}
\subsection{Instruction Layout}
\subsection{Addressing Modes}
\section{Instruction Set}
\section{Interrupt Model}
\section{Memory Management Model}

\chapter{Register Set}
This chapter describes the registers. The registers are seperated in four groups based on accessability.
\section{Foreground Registers}
The foreground registers are the registers all regular instructions can read from and write to. There are eight 8-bit and eight 16-bit foreground registers. These registers are preserved in interrupts.
\begin{table}[]
\centering
\caption{List of Foreground Registers}
\label{foreground-registers-list}
\begin{tabular}{cclc}
\hiderowcolors
\textbf{ID}  & \textbf{Mnonic} & \textbf{Descriptive Name} & \textbf{Length in bits} \\ \hline
\showrowcolors
0x0 & al & The lower byte of ax  & 8  \\
0x1 & ah & The higher byte of ax & 8  \\
0x2 & bl & The lower byte of bx  & 8  \\
0x3 & bh & The higher byte of bx & 8  \\
0x4 & cl & The lower byte of cx  & 8  \\
0x5 & ch & The higher byte of cx & 8  \\
0x6 & dl & The lower byte of dx  & 8  \\
0x7 & dh & The higher byte of dx & 8  \\
0x8 & ax & The first GPR         & 16 \\
0x9 & bx & The second GPR        & 16 \\
0xA & cx & The third GPR         & 16 \\
0xB & dx & The fourth GPR        & 16 \\
0xC & ex & The fifth GPR         & 16 \\
0xD & tm & Temporary data        & 16 \\
0xE & sp & Stack pointer         & 16 \\
0xF & pc & Program counter       & 16 \\
\end{tabular}
\end{table}
\subsection{General-Purpose Registers (GPRs)}
These instruction
\subsection{Stack Pointer Register (SP)}
\subsection{Program Counter Register (PC)}
\section{Background Registers}
\subsection{Backup Registers}
\subsection{Interrupt Registers}
\section{Indirect Registers}
\subsection{Flags Register}
\subsection{Section Registers}
\section{Output Registers}

\chapter{Operand Conventions and Addressing Modes}
\section{Operand Conventions}
\subsection{Bit and Byte Ordering}
\subsection{Aligned and Misaligned Memory Access}
\section{Addressing Modes}
\subsection{Not From Memory}
\subsubsection{Register Direct}
\subsubsection{Absolute}
\subsubsection{Register with displacement}
\subsection{From Memory}
\subsubsection{Direct}
\subsubsection{Base Plus Displacement}

\chapter{Instruction Set Summary}
\section{Instruction Types}
\subsection{Move Instructions}
\subsection{Arithimic Instructions}
\subsection{Control Instructions}
\subsection{Reserved Instructions}
\subsection{Artifact Instructions}
\section{Instruction Format}
\subsection{exceptions}

\chapter{Memory Management}
\section{Segments}
\subsection{Base}
\subsection{Limit}
\section{Code Segment (CS)}
\section{Data Segment (DS)}
\section{Segment Switching}
\section{Changing Segments}

\chapter{Interrupts}
\section{Interrupt Enabling/Disabling}
\subsection{Interrupt Enabled Flag}
\subsection{Internal Interrupt Mask}
\subsection{Programmable Interrupt Controller (PIC)}
\section{State Preservation}
\subsection{Backup}
\subsection{Full Restore}
\subsection{Partial Restore}
\section{Interrupt Service Routine (ISR)}
\subsection{Interrupt Far Jump}
\subsection{Return to Context}
\subsection{Switch Context}

\chapter{I/O Conventions}
\section{Reading Input}
\section{Writing Output}
\subsection{Problem with Interrupts}

\chapter{Instruction Set}
\section*{0x00: MOVZ}
\section*{0x01: MOVNZ}
\section*{0x02: MOVS}
\section*{0x03: MOVNS}
\section*{...}

% Appendix
\appendix
%\backmatter
\chapter{Instruction Set Listings}

\begin{center}
\begin{longtable}{cccccc}
\caption{List of instructions sorted by opcode} 
\label{opcode_sorted_instructions_list} \\
\hiderowcolors
\multicolumn{3}{c}{\textbf{Opcode}} & \multirow{2}{*}{\textbf{Memnonic}} & \multirow{2}{*}{\textbf{Operand A}} & \multirow{2}{*}{\textbf{Operand B}} \\
\textbf{Decimal} & \textbf{Hex} & \textbf{Binary} &  &  &  \\ \hline 
\showrowcolors 
\endhead
0  & 0x00 & 000000 & MOV   & R\&W  & R  \\
1  & 0x01 & 000001 & MOV   & R\&W  & R  \\
2  & 0x02 & 000010 & MOV   & R\&W  & R  \\
3  & 0x03 & 000011 & MOV   & R\&W  & R  \\
4  & 0x04 & 000100 & MOV   & R\&W  & R  \\
5  & 0x05 & 000101 & MOV   & R\&W  & R  \\
6  & 0x06 & 000110 & MOV   & R\&W  & R  \\
7  & 0x07 & 000111 & MOV   & R\&W  & R  \\
8  & 0x08 & 001000 & MOV   & R\&W  & R  \\
9  & 0x09 & 001001 & MOV   & R\&W  & R  \\
10 & 0x0A & 001010 & MOV   & R\&W  & R  \\
11 & 0x0B & 001011 & MOV   & R\&W  & R  \\
12 & 0x0C & 001100 & MOV   & R\&W  & R  \\
13 & 0x0D & 001101 & MOV   & R\&W  & R  \\
14 & 0x0E & 001110 & MOV   & R\&W  & R  \\
15 & 0x0F & 001111 & MOV   & R\&W  & R  \\
16 & 0x10 & 010000 & MOV   & R\&W  & R  \\
17 & 0x11 & 010001 & MOV   & R\&W  & R  \\
18 & 0x12 & 010010 & MOV   & R\&W  & R  \\
19 & 0x13 & 010011 & MOV   & R\&W  & R  \\
20 & 0x14 & 010100 & MOV   & R\&W  & R  \\
21 & 0x15 & 010101 & MOV   & R\&W  & R  \\
22 & 0x16 & 010110 & MOV   & R\&W  & R  \\
23 & 0x17 & 010111 & MOV   & R\&W  & R  \\
24 & 0x18 & 011000 & MOV   & R\&W  & R  \\
25 & 0x19 & 011001 & MOV   & R\&W  & R  \\
26 & 0x1A & 011010 & MOV   & R\&W  & R  \\
27 & 0x1B & 011011 & MOV   & R\&W  & R  \\
28 & 0x1C & 011100 & MOV   & R\&W  & R  \\
29 & 0x1D & 011101 & MOV   & R\&W  & R  \\
30 & 0x1E & 011110 & MOV   & R\&W  & R  \\
31 & 0x1F & 011111 & MOV   & R\&W  & R  \\
32 & 0x20 & 100000 & MOV   & R\&W  & R  \\
33 & 0x21 & 100001 & MOV   & R\&W  & R  \\
34 & 0x22 & 100010 & MOV   & R\&W  & R  \\
35 & 0x23 & 100011 & MOV   & R\&W  & R  \\
36 & 0x24 & 100100 & MOV   & R\&W  & R  \\
37 & 0x25 & 100101 & MOV   & R\&W  & R  \\
38 & 0x26 & 100110 & MOV   & R\&W  & R  \\
39 & 0x27 & 100111 & MOV   & R\&W  & R  \\
40 & 0x28 & 101000 & MOV   & R\&W  & R  \\
41 & 0x29 & 101001 & MOV   & R\&W  & R  \\
42 & 0x2A & 101010 & MOV   & R\&W  & R  \\
43 & 0x2B & 101011 & MOV   & R\&W  & R  \\
44 & 0x2C & 101100 & MOV   & R\&W  & R  \\
45 & 0x2D & 101101 & MOV   & R\&W  & R  \\
46 & 0x2E & 101110 & MOV   & R\&W  & R  \\
47 & 0x2F & 101111 & MOV   & R\&W  & R  \\
48 & 0x30 & 110000 & MOV   & R\&W  & R  \\
49 & 0x31 & 110001 & MOV   & R\&W  & R  \\
50 & 0x32 & 110010 & MOV   & R\&W  & R  \\
51 & 0x33 & 110011 & MOV   & R\&W  & R  \\
52 & 0x34 & 110100 & MOV   & R\&W  & R  \\
53 & 0x35 & 110101 & MOV   & R\&W  & R  \\
54 & 0x36 & 110110 & MOV   & R\&W  & R  \\
55 & 0x37 & 110111 & MOV   & R\&W  & R  \\
56 & 0x38 & 111000 & MOV   & R\&W  & R  \\
57 & 0x39 & 111001 & MOV   & R\&W  & R  \\
58 & 0x3A & 111010 & MOV   & R\&W  & R  \\
59 & 0x3B & 111011 & MOV   & R\&W  & R  \\
60 & 0x3C & 111100 & MOV   & R\&W  & R  \\
61 & 0x3D & 111101 & MOV   & R\&W  & R  \\
62 & 0x3E & 111110 & MOV   & R\&W  & R  \\
63 & 0x3F & 111111 & MOV   & R\&W  & R  \\
\end{longtable}
\end{center}

\chapter{Simplified Mnemonics}

\chapter{Common Procedures}

\chapter{Standard Peripherals}
\section{Programmable Interrupt Controller (PIC)}
\section{Keyboard}
\section{Programmable Interrupt Timer (PIT)}
\section{Sound Card}
\section{Graphical Card}
\section{Memory Control Hub (MCH)}
\section{Segements and Out Of Bounds Exception}

\end{document}
